\hypertarget{index_intro_sec}{}\section{Introduction}\label{index_intro_sec}
Start at {\ttfamily Files $>$ Files List} to view functions and their documentaion\hypertarget{index_Writing}{}\section{Custom Code}\label{index_Writing}
Two timers (Timer 6 and Timer 7) are set up in the code and can execute at user-\/set frequencies in {\ttfamily \mbox{\hyperlink{r2r_8c}{r2r.\+c}}}. The intention is that the user will either write code in \mbox{\hyperlink{r2r_8c}{r2r.\+c}} or create a header and source file. Here, an simple example will be set out to compute simple position control over motor 1 in \mbox{\hyperlink{r2r_8c}{r2r.\+c}} at 5k\+Hz.

In {\ttfamily \mbox{\hyperlink{r2r_8c}{r2r.\+c}}}, update frequency\+: \begin{DoxyVerb}    #define TIMER_6_FREQUENCY 5000
\end{DoxyVerb}


Under {\ttfamily \mbox{\hyperlink{r2r_8c_a57b21594b75d4b2a140a1f9bbb1465e8}{T\+I\+M\+E\+R6\+Int\+Handler()}}}\+: \begin{DoxyVerb}void TIMER6IntHandler(){
    float kp = 0.5;
    float ki = 1;
    float kd = 0;
        desired_position = 60; //degrees
    encoderRead(); // update the encoder
    int actual = readMotor1RawRelative(); // read counts set to relative
    int raw = readMotor1Raw(); // read counts absolute
    int error = angles2Counts(desired_position) - actual;
    int u = kp*error+kd*prev_error+ki*int_error;
    motor1ControlPWM(u);
 }  \end{DoxyVerb}
 